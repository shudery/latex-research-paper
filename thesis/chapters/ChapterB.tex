% !Mode:: "TeX:UTF-8"

\chapter{时域积分方程基础}
时域积分方程(TDIE)方法作为分析瞬态电磁波动现象最主要的数值算法之
一,常用于求解均匀散射体和表面散射体的瞬态电磁散射问题。
\section{时域积分方程的类型}
\section{空间基函数与时间基函数}
利用数值算法求解时域积分方程,首先需要选取适当的空间基函数与时间基
函数对待求感应电流进行离散\footnote{脚注序号“①,……,⑩”的字体是“正文”,不是“上标”,序号与脚注内容文字之间空1个半角字符,脚注的段落格式为:单倍行距,段前空0磅,段后空0磅,悬挂缩进1.5字符;中文用宋体,字号为小五号,英文和数字用Times New Roman字体,字号为9磅;中英文混排时,所有标点符号(例如逗号“,”、括号“()”等)一律使用中文输入状态下的标点符号,但小数点采用英文状态下的样式“.”。}。
\subsection{空间基函数}
……

RWG 基函数是定义在三角形单元上的最具代表性的基函数。它的具体定义如
下:
\begin{equation}
f_n(r)=
\begin{cases}
\frac{l_n}{2A_n^+}\rho_n^+=\frac{l_n}{2A_n^+}(r-r_+)&r\in T_n^+\\
\frac{l_n}{2A_n^-}\rho_n^-=\frac{l_n}{2A_n^-}(r_--r)&r\in T_n^-\\
0&\text{其它}\\
\end{cases}
\end{equation}
其中,$l_n$为三角形单元$T_n^+$和$T_n^-$公共边的长度,$A_n^+$和$A_n^-$分别为三角形单元$T_n^+$和$T_n^-$的面积(如图\ref{pica}所示)。
\pic[h]{RWG 基函数几何参数示意图}{}{pica}

……

\subsection{时间基函数}
……
\subsubsection{时域方法特有的展开函数}
……
\subsubsection{频域方法特有的展开函数}
……
\section{入射波}
……

如图\ref{picb}和图\ref{picc}所示分别给出了参数$E_0=\hat{x}$,$a_n=-\hat{z}$,$f_0=250MHz$,$f_w=50MHz$,$t_w=4.2\sigma$时,调制高斯脉冲的时域与频域归一化波形图。
\begin{pics}[h]{调制高斯脉冲时域与频率波形}{picbc}
\addsubpic{调制高斯脉冲时域波形}{keepaspectratio=false,height=5.95cm,width=7.3cm}{picb}
\addsubpic{调制高斯脉冲频域波形}{keepaspectratio=false,height=5.65cm,width=6.41cm}{picc}
\end{pics}
\section{本章小结}
本章首先从时域麦克斯韦方程组出发推导得到了时域电场、磁场以及混合场
积分方程。……
